\documentclass[11pt, a4paper]{article}

% --- Basic Formatting Packages ---
% Note: utf8 inputenc might not work on very old systems. 
% If you encounter errors, remove the inputenc line.
\usepackage[utf8]{inputenc} 
\usepackage[T1]{fontenc}
\usepackage{geometry}
\geometry{margin=1in}
\usepackage{url}

% --- Graphics Package ---
% 'graphicx' is required to insert the EPS file generated by Gnuplot
\usepackage{graphicx}

% Note: Modern packages like 'pgfplots' and 'biblatex' are unavailable
% in this environment. We revert to standard BibTeX and EPS inclusion.

\title{\textbf{Introduction to the Debian System}}
\author{Debian User}
\date{\today}

\begin{document}

\maketitle

\section{Overview}

Debian is one of the oldest and most influential Linux distributions in existence. Founded by Ian Murdock in 1993, it was named after his girlfriend (later wife) Debra and himself. Unlike many other operating systems, Debian is not developed by a single company but by a massive, distributed community of volunteers known as the Debian Project.

The system is renowned for its stability, its strict adherence to the Free Software philosophy, and its robust package management system, \texttt{APT} (Advanced Package Tool). Debian serves as the upstream foundation for numerous other popular distributions, including Ubuntu, Kali Linux, and Linux Mint.

The project is guided by the \textit{Debian Social Contract}, which ensures the system remains 100\% free software. Over the decades, the complexity and utility of the OS have grown exponentially, represented by the size of its software repository \cite{debianhistory}.

\section{Repository Growth}

One of the primary metrics for the success and scale of Debian is the number of software packages available in its official repositories. Starting from a modest collection in the mid-90s, the repository has grown to include tens of thousands of pre-compiled binaries.

Figure \ref{fig:pkg_growth} illustrates the approximate number of packages included in major Debian stable releases, from Debian 0.93 (1995) to Debian 13 "Trixie" (2025).

% --- Insert the EPS Figure ---
\begin{figure}[h]
    \centering
    % Include the .eps file generated by gnuplot
    % The 'width' parameter scales the image to fit the page text width
    \includegraphics[width=0.9\textwidth]{debian_plot.eps}
    \caption{The trajectory of available software packages in Debian stable releases over time.}
    \label{fig:pkg_growth}
\end{figure}

% --- Bibliography Setup ---
% Use the standard 'plain' style for BibTeX
\bibliographystyle{plain}
% Point to the external references.bib file
\bibliography{references} 

\end{document}